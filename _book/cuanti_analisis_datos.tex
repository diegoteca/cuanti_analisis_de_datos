% Options for packages loaded elsewhere
\PassOptionsToPackage{unicode}{hyperref}
\PassOptionsToPackage{hyphens}{url}
%
\documentclass[
]{book}
\usepackage{amsmath,amssymb}
\usepackage{lmodern}
\usepackage{iftex}
\ifPDFTeX
  \usepackage[T1]{fontenc}
  \usepackage[utf8]{inputenc}
  \usepackage{textcomp} % provide euro and other symbols
\else % if luatex or xetex
  \usepackage{unicode-math}
  \defaultfontfeatures{Scale=MatchLowercase}
  \defaultfontfeatures[\rmfamily]{Ligatures=TeX,Scale=1}
\fi
% Use upquote if available, for straight quotes in verbatim environments
\IfFileExists{upquote.sty}{\usepackage{upquote}}{}
\IfFileExists{microtype.sty}{% use microtype if available
  \usepackage[]{microtype}
  \UseMicrotypeSet[protrusion]{basicmath} % disable protrusion for tt fonts
}{}
\makeatletter
\@ifundefined{KOMAClassName}{% if non-KOMA class
  \IfFileExists{parskip.sty}{%
    \usepackage{parskip}
  }{% else
    \setlength{\parindent}{0pt}
    \setlength{\parskip}{6pt plus 2pt minus 1pt}}
}{% if KOMA class
  \KOMAoptions{parskip=half}}
\makeatother
\usepackage{xcolor}
\IfFileExists{xurl.sty}{\usepackage{xurl}}{} % add URL line breaks if available
\IfFileExists{bookmark.sty}{\usepackage{bookmark}}{\usepackage{hyperref}}
\hypersetup{
  pdftitle={Diego Armando Datos},
  pdfauthor={Metodistas del Sur},
  hidelinks,
  pdfcreator={LaTeX via pandoc}}
\urlstyle{same} % disable monospaced font for URLs
\usepackage{color}
\usepackage{fancyvrb}
\newcommand{\VerbBar}{|}
\newcommand{\VERB}{\Verb[commandchars=\\\{\}]}
\DefineVerbatimEnvironment{Highlighting}{Verbatim}{commandchars=\\\{\}}
% Add ',fontsize=\small' for more characters per line
\usepackage{framed}
\definecolor{shadecolor}{RGB}{248,248,248}
\newenvironment{Shaded}{\begin{snugshade}}{\end{snugshade}}
\newcommand{\AlertTok}[1]{\textcolor[rgb]{0.94,0.16,0.16}{#1}}
\newcommand{\AnnotationTok}[1]{\textcolor[rgb]{0.56,0.35,0.01}{\textbf{\textit{#1}}}}
\newcommand{\AttributeTok}[1]{\textcolor[rgb]{0.77,0.63,0.00}{#1}}
\newcommand{\BaseNTok}[1]{\textcolor[rgb]{0.00,0.00,0.81}{#1}}
\newcommand{\BuiltInTok}[1]{#1}
\newcommand{\CharTok}[1]{\textcolor[rgb]{0.31,0.60,0.02}{#1}}
\newcommand{\CommentTok}[1]{\textcolor[rgb]{0.56,0.35,0.01}{\textit{#1}}}
\newcommand{\CommentVarTok}[1]{\textcolor[rgb]{0.56,0.35,0.01}{\textbf{\textit{#1}}}}
\newcommand{\ConstantTok}[1]{\textcolor[rgb]{0.00,0.00,0.00}{#1}}
\newcommand{\ControlFlowTok}[1]{\textcolor[rgb]{0.13,0.29,0.53}{\textbf{#1}}}
\newcommand{\DataTypeTok}[1]{\textcolor[rgb]{0.13,0.29,0.53}{#1}}
\newcommand{\DecValTok}[1]{\textcolor[rgb]{0.00,0.00,0.81}{#1}}
\newcommand{\DocumentationTok}[1]{\textcolor[rgb]{0.56,0.35,0.01}{\textbf{\textit{#1}}}}
\newcommand{\ErrorTok}[1]{\textcolor[rgb]{0.64,0.00,0.00}{\textbf{#1}}}
\newcommand{\ExtensionTok}[1]{#1}
\newcommand{\FloatTok}[1]{\textcolor[rgb]{0.00,0.00,0.81}{#1}}
\newcommand{\FunctionTok}[1]{\textcolor[rgb]{0.00,0.00,0.00}{#1}}
\newcommand{\ImportTok}[1]{#1}
\newcommand{\InformationTok}[1]{\textcolor[rgb]{0.56,0.35,0.01}{\textbf{\textit{#1}}}}
\newcommand{\KeywordTok}[1]{\textcolor[rgb]{0.13,0.29,0.53}{\textbf{#1}}}
\newcommand{\NormalTok}[1]{#1}
\newcommand{\OperatorTok}[1]{\textcolor[rgb]{0.81,0.36,0.00}{\textbf{#1}}}
\newcommand{\OtherTok}[1]{\textcolor[rgb]{0.56,0.35,0.01}{#1}}
\newcommand{\PreprocessorTok}[1]{\textcolor[rgb]{0.56,0.35,0.01}{\textit{#1}}}
\newcommand{\RegionMarkerTok}[1]{#1}
\newcommand{\SpecialCharTok}[1]{\textcolor[rgb]{0.00,0.00,0.00}{#1}}
\newcommand{\SpecialStringTok}[1]{\textcolor[rgb]{0.31,0.60,0.02}{#1}}
\newcommand{\StringTok}[1]{\textcolor[rgb]{0.31,0.60,0.02}{#1}}
\newcommand{\VariableTok}[1]{\textcolor[rgb]{0.00,0.00,0.00}{#1}}
\newcommand{\VerbatimStringTok}[1]{\textcolor[rgb]{0.31,0.60,0.02}{#1}}
\newcommand{\WarningTok}[1]{\textcolor[rgb]{0.56,0.35,0.01}{\textbf{\textit{#1}}}}
\usepackage{longtable,booktabs,array}
\usepackage{calc} % for calculating minipage widths
% Correct order of tables after \paragraph or \subparagraph
\usepackage{etoolbox}
\makeatletter
\patchcmd\longtable{\par}{\if@noskipsec\mbox{}\fi\par}{}{}
\makeatother
% Allow footnotes in longtable head/foot
\IfFileExists{footnotehyper.sty}{\usepackage{footnotehyper}}{\usepackage{footnote}}
\makesavenoteenv{longtable}
\usepackage{graphicx}
\makeatletter
\def\maxwidth{\ifdim\Gin@nat@width>\linewidth\linewidth\else\Gin@nat@width\fi}
\def\maxheight{\ifdim\Gin@nat@height>\textheight\textheight\else\Gin@nat@height\fi}
\makeatother
% Scale images if necessary, so that they will not overflow the page
% margins by default, and it is still possible to overwrite the defaults
% using explicit options in \includegraphics[width, height, ...]{}
\setkeys{Gin}{width=\maxwidth,height=\maxheight,keepaspectratio}
% Set default figure placement to htbp
\makeatletter
\def\fps@figure{htbp}
\makeatother
\setlength{\emergencystretch}{3em} % prevent overfull lines
\providecommand{\tightlist}{%
  \setlength{\itemsep}{0pt}\setlength{\parskip}{0pt}}
\setcounter{secnumdepth}{5}
\usepackage{booktabs}
\ifLuaTeX
  \usepackage{selnolig}  % disable illegal ligatures
\fi
\usepackage[]{natbib}
\bibliographystyle{apalike}

\title{Diego Armando Datos}
\author{Metodistas del Sur}
\date{2022-03-09}

\usepackage{amsthm}
\newtheorem{theorem}{Teorema}[chapter]
\newtheorem{lemma}{Lema}[chapter]
\newtheorem{corollary}{Corolario}[chapter]
\newtheorem{proposition}{Proposición}[chapter]
\newtheorem{conjecture}{Conjecture}[chapter]
\theoremstyle{definition}
\newtheorem{definition}{Definición}[chapter]
\theoremstyle{definition}
\newtheorem{example}{Ejemplo}[chapter]
\theoremstyle{definition}
\newtheorem{exercise}{Ejercicio}[chapter]
\theoremstyle{definition}
\newtheorem{hypothesis}{Hypothesis}[chapter]
\theoremstyle{remark}
\newtheorem*{remark}{Nota: }
\newtheorem*{solution}{Solución }
\begin{document}
\maketitle

{
\setcounter{tocdepth}{1}
\tableofcontents
}
\hypertarget{presentaciuxf3n-y-objetivos-del-informe}{%
\chapter*{Presentación y objetivos del informe}\label{presentaciuxf3n-y-objetivos-del-informe}}
\addcontentsline{toc}{chapter}{Presentación y objetivos del informe}

El siguiente informe tiene una finalidad principalmente pedagógica. Específicamente está pensado como un mecanismo para ayudar a alcanzar los siguientes objetivos que forman parte del \href{https://docs.google.com/document/d/15ZuHJ1ZM7Z0g0Edt-mv1PCB697-x6-rZfcWdAtd85yM/edit\#heading=h.s43n504lcmmx}{programa} de la materia ``Metodología y Técnicas Cuantitativas'' de la UNAJ:

\begin{itemize}
\item
  Lograr un conocimiento mínimo de la existencia y pertinencia de las técnicas de análisis de datos \emph{básicas} y una \emph{habilidad} mínima en la ejecución de las mismas.
\item
  \emph{Lograr un} conocimiento mínimo de la existencia y pertinencia de (otras) técnicas de análisis de datos más específicas y (usualmente) más \emph{complejas}.
\end{itemize}

Aparte de la distinción entre técnicas básicas y complejas la diferencia en cuanto a los objetivos pedagógicos es que entre ambos objetivos es que las últimas sólo se aspira a conocerlas mientras que en las primeras se espera, también, que se adquiera cierta habilidad en su ejecución. Expresado de otro modo, de las técnicas consideradas complejas se espera que se sepa de su existencia y qué tipos de problemas ayuda a solucionar. En cambio, en las técnicas básicas se espera lo anterior pero que además se adquiera la habilidad de poder ejecutar las mismas.

Para lograr lo anterior durante la cursada de la materia se realiza una encuesta mediante un \href{https://www.google.com/intl/es-419_ar/forms/about/}{formulario de google} que contestan los mismos estudiantes. Luego el contenido de sus respuestas se analiza con el \href{https://www.r-project.org/}{programa R} y con el mismo programa se escribe y publica este informe.

En cuanto al estilo del texto será deliberamente informal aunque con 3 excepciones con fines principalmente pedagógicos:

\begin{itemize}
\item
  Se incluirán piezas de código
\item
  Se hará uso de citas y referencias (estilo APA)
\item
  Los conceptos importantes apareceran traducidos al ingles entre paréntesis.
\end{itemize}

\hypertarget{diseuxf1o-de-la-encuesta}{%
\chapter{Diseño de la encuesta}\label{diseuxf1o-de-la-encuesta}}

En el diseño de la encuesta que contestan los estudiantes se pueden distinguir 2 grandes procesos. Uno referido al diseño del cuestionario y otro referido al diseño de la selección de los casos. El primero se preocupa por \textbf{qué se pregunta} y el segundo por \textbf{a quienes}. En esta sección la mayoría de las veces sólo se explicitará pero pocas veces se justificará las decisiones tomadas en ambos procesos. En otras palabras, la palabra ``diseño'' le queda grande a ambos procesos para esta encuesta. Lamentablemente lo mismo puede afirmarse de varias encuestas académicas o profesionales. La justificación particular para la falta de atención metodológica a estos puntos es que en este caso se trata de una actividad con fines principalmente pedagógicos más que académicos o profesionales.

\hypertarget{diseuxf1o-del-cuestionario}{%
\section{Diseño del cuestionario}\label{diseuxf1o-del-cuestionario}}

El \href{https://drive.google.com/file/d/1nbU16-b2RxvPQZV2SS-Aodl1d7zPr9Cw/view?usp=sharing}{cuestionario}, como muchos cuestionarios que se usan en la práctica profesional, está diseñado con la lógica de módulos. Como veremos más adelante en esta instancia cada módulo tiene una justificable lógica interna pero lo que no tiene este cuestionario es una coherencia interna que explique la relación entre los diferentes módulos.

El diseño de un cuestionario mediante la estrategia de módulos suele ser útil tanto para el encuestado como para el investigador. Al encuestado lo ayuda a ordenarse visualmente especialmente si se trata de un cuestionario largo. Al investigador lo ayuda tanto para el etiquetamiento (o nombramiento) como para el orden de las variables en la base de datos.

El cuestionario por ahora cuenta con los siguientes módulos:

\begin{enumerate}
\def\labelenumi{\arabic{enumi}.}
\item
  Identificación
\item
  Demográfico
\item
  Composición del hogar
\item
  Cuidados
\item
  Vivienda
\item
  Uso del tiempo
\item
  Inseguridad alimentaria
\item
  Preferencias sociales
\item
  Origen social
\item
  Trabajo actual
\item
  Ingresos del hogar
\item
  Académico UNAJ
\item
  Expectativas materia cuantitativa
\item
  Redes sociales entre estudiantes de la materia
\end{enumerate}

Claramente no parece haber mucha relación entre los diferentes módulos. A diferencia de algunas encuestas ómnibus en donde este problema se presenta porque cada investigador que forma parte de la investigación agrega su propio ``paquete de preguntas'' para su particular investigación en este caso la justificación es que cada módulo tiene una función pedagógica particular.\footnote{En una investigación ómnibus varios investigadores (o clientes) comparten el mismo diseño de selección de casos y de esa manera, juntos, pueden llegar a más casos ya que entre todos amortizan estos costos que suelen ser grandes en encuestas grandes presenciales. A cambio, más allá de un algunos módulos comunes (p.e. el demográfico), cada investigador o cliente agrega su módulo de particular interés.}

\hypertarget{diseuxf1o-de-la-selecciuxf3n-de-casos}{%
\section{Diseño de la selección de casos}\label{diseuxf1o-de-la-selecciuxf3n-de-casos}}

En cuanto al criterio utilizado para la selección de los casos se puede afirmar que se trata de una muestra de los estudiantes de la materia ``Metodología y Técnicas Cuantitativas'' de la carrera de Trabajo Social de la Universidad Nacional Arturo Jauretche. Esto ya afirma algo pero se puede especificar aún más.

Antes que nada, si bien es una encuesta que potencialmente les llega a todos los estudiantes que cursan la materia, la misma efectivamente sólo llega a una muestra de los mismos. Esto se produce principalmente por la no-respuesta de algunos estudiantes que comienzan la cursada pero no responden la encuesta. El factor ``deserción'' no parece afectar tanto a la selección de casos dado que la encuesta se realiza al principio de la cursada. Esto hace que la encuesta, si es representativa de algo, lo sea de la población intermedia que queda conformada entre los inscriptos a la materia y los que finalmente la regularizan. Obviamente esto no alcanza para decir mucho acerca de la representatividad de la encuesta sobre poblaciones mayores compuestas por:

\begin{itemize}
\item
  Los estudiantes de la carrera de Trabajo Social de la UNAJ.
\item
  Los estudiantes de la UNAJ
\item
  Los estudiantes del sistema universitario nacional
\end{itemize}

\hypertarget{cross}{%
\chapter{Limpieza y consistencia de los datos}\label{cross}}

El producto de los procesos de producción y registro de los datos es, precisamente, un dato. Este dato puede (y suele) contener diferentes tipos de errores por lo que se considera una buena práctica realizar un proceso de limpieza y preparación para recién después comenzar el proceso estricto del análisis de los datos.

En esta sección veremos algunos ejemplos tanto de limpieza, consistencia y construcción de nuevas variables. Aquí veremos ejemplos de los casos mas sencillos. Procesos como el pegado (\emph{joint}) de variables, necesario cuando los datos se encuentran en diferentes archivos, no se verán.

\hypertarget{limpieza}{%
\section{Limpieza}\label{limpieza}}

La idea de limpieza (\emph{cleaning}) viene de usar la metáfora de dato sucio (\emph{dirty}). Un dato sucio no necesariamente es un dato incorrecto aunque sí se trata de un tipo de dato incómodo ya que dificulta el posterior análisis.

La tarea básica de limpieza (\emph{cleaning}) que se hará será el remombre de todas las variables. La razón de esta operación es que, al menos si se trabaja con google forms, los nombres de las variables son el texto de la propia pregunta del formulario. Esto incomoda un poco el análisis de los datos por la gran extensión de algunas preguntas y por lo problemático que a veces puede ser tener espacios en blanco en los nombres de las variables.

\begin{verbatim}
::: {.rmdcaution}
The `bs4_book` style also includes an `.rmdnote` callout block
like this one.
:::
\end{verbatim}

bdfb

\begin{verbatim}
 ::: {.rmdimportant}
The `bs4_book` style also includes an `.rmdnote` callout block
like this one.
:::
\end{verbatim}

holis

\begin{verbatim}
 ::: {.rmdtip}
The `bs4_book` style also includes an `.rmdnote` callout block
like this one.
:::
\end{verbatim}

holis

\begin{verbatim}
 ::: {.rmdwarning}
The `bs4_book` style also includes an `.rmdnote` callout block
like this one.
:::
\end{verbatim}

holis

\begin{verbatim}
 ::: {.rmdnote}
The `bs4_book` style also includes an `.rmdnote` callout block
like this one.
:::
\end{verbatim}

holis

\begin{verbatim}
::: {.rmdnote}
The `bs4_book` style also includes an `.rmdnote` callout block
like this one.

:::
\end{verbatim}

\begin{Shaded}
\begin{Highlighting}[]
\CommentTok{\# El objetivo de este script es descargar las respuestas del cuestionario}

\CommentTok{\# Librerías {-}{-}{-}{-}}

\FunctionTok{library}\NormalTok{(tidyverse) }\CommentTok{\# Uso general}
\CommentTok{\#\textgreater{} {-}{-} Attaching packages {-}{-}{-}{-}{-}{-}{-}{-}{-}{-}{-}{-}{-}{-}{-}{-}{-}{-}{-} tidyverse 1.3.1 {-}{-}}
\CommentTok{\#\textgreater{} v ggplot2 3.3.5     v purrr   0.3.4}
\CommentTok{\#\textgreater{} v tibble  3.1.6     v dplyr   1.0.8}
\CommentTok{\#\textgreater{} v tidyr   1.2.0     v stringr 1.4.0}
\CommentTok{\#\textgreater{} v readr   2.1.2     v forcats 0.5.1}
\CommentTok{\#\textgreater{} {-}{-} Conflicts {-}{-}{-}{-}{-}{-}{-}{-}{-}{-}{-}{-}{-}{-}{-}{-}{-}{-}{-}{-}{-}{-} tidyverse\_conflicts() {-}{-}}
\CommentTok{\#\textgreater{} x dplyr::filter() masks stats::filter()}
\CommentTok{\#\textgreater{} x dplyr::lag()    masks stats::lag()}
\FunctionTok{library}\NormalTok{(here) }\CommentTok{\# Mejora la replicabilidad en diferentes computadoras}
\CommentTok{\#\textgreater{} here() starts at C:/Users/dquar/OneDrive/Documentos/R{-}Studio Projects/cuanti\_analisis\_de\_datos}
\FunctionTok{library}\NormalTok{(googlesheets4) }\CommentTok{\# Específico para googlesheet}
\FunctionTok{library}\NormalTok{(googledrive) }\CommentTok{\# Más general para googledrive}
\CommentTok{\#\textgreater{} }
\CommentTok{\#\textgreater{} Attaching package: \textquotesingle{}googledrive\textquotesingle{}}
\CommentTok{\#\textgreater{} The following objects are masked from \textquotesingle{}package:googlesheets4\textquotesingle{}:}
\CommentTok{\#\textgreater{} }
\CommentTok{\#\textgreater{}     request\_generate, request\_make}
\FunctionTok{library}\NormalTok{(janitor) }\CommentTok{\# Específico para limpieza y exploración}
\CommentTok{\#\textgreater{} }
\CommentTok{\#\textgreater{} Attaching package: \textquotesingle{}janitor\textquotesingle{}}
\CommentTok{\#\textgreater{} The following objects are masked from \textquotesingle{}package:stats\textquotesingle{}:}
\CommentTok{\#\textgreater{} }
\CommentTok{\#\textgreater{}     chisq.test, fisher.test}
\FunctionTok{library}\NormalTok{(lubridate) }\CommentTok{\# Manejo de tiempos fechas}
\CommentTok{\#\textgreater{} }
\CommentTok{\#\textgreater{} Attaching package: \textquotesingle{}lubridate\textquotesingle{}}
\CommentTok{\#\textgreater{} The following objects are masked from \textquotesingle{}package:base\textquotesingle{}:}
\CommentTok{\#\textgreater{} }
\CommentTok{\#\textgreater{}     date, intersect, setdiff, union}
\FunctionTok{library}\NormalTok{(readxl) }\CommentTok{\# Lectura de archivos excel}

\CommentTok{\# Identificación {-}{-}{-}{-}}

\FunctionTok{i\_am}\NormalTok{(}\StringTok{"scripts/cleaning.r"}\NormalTok{)}
\CommentTok{\#\textgreater{} here() starts at C:/Users/dquar/OneDrive/Documentos/R{-}Studio Projects/cuanti\_analisis\_de\_datos}

\CommentTok{\# Identificación google drive}

\FunctionTok{drive\_auth}\NormalTok{(}\AttributeTok{email =} \StringTok{"dquartullidocencia@gmail.com"}\NormalTok{)}

\CommentTok{\# Link del cuestionario editable (sólo para tener a mano)}

\CommentTok{\# https://docs.google.com/forms/u/1/d/13mj2wN16HMaKtK0b0rbKEH4YyabCYZyO9P0qCoplisQ/edit?usp=send\_form\&usp=redirect\_edit\_m2}

\CommentTok{\# Descarga de archivo online {-}{-}{-}{-}}

\NormalTok{url }\OtherTok{\textless{}{-}} \StringTok{"https://docs.google.com/spreadsheets/d/1kVxZnwLGlkqMSWvsO5G93EfujElhP8ZToJs{-}ki0rhwo/edit?resourcekey\#gid=1385801546"}

\CommentTok{\# Lo grabo en el equipo como copia de respaldo}

\FunctionTok{drive\_download}\NormalTok{(url,}
               \FunctionTok{here}\NormalTok{(}\StringTok{"Inputs"}\NormalTok{, }\StringTok{"descarga\_original.xlsx"}\NormalTok{),}
               \AttributeTok{overwrite =} \ConstantTok{TRUE}\NormalTok{)}
\CommentTok{\#\textgreater{} File downloaded:}
\CommentTok{\#\textgreater{} * \textquotesingle{}Cuestionario Cuanti UNAJ (respuestas)\textquotesingle{}}
\CommentTok{\#\textgreater{}   \textless{}id: 1kVxZnwLGlkqMSWvsO5G93EfujElhP8ZToJs{-}ki0rhwo\textgreater{}}
\CommentTok{\#\textgreater{} Saved locally as:}
\CommentTok{\#\textgreater{} * \textquotesingle{}C:/Users/dquar/OneDrive/Documentos/R{-}Studio Projects/cuanti\_analisis\_de\_datos/Inputs/descarga\_original.xlsx\textquotesingle{}}

\CommentTok{\# Leo el archivo online y comienzo la corrección de los nombres de las variables}
                    
\NormalTok{df\_encuesta }\OtherTok{\textless{}{-}} \FunctionTok{read\_xlsx}\NormalTok{(}\FunctionTok{here}\NormalTok{(}\StringTok{"Inputs"}\NormalTok{, }\StringTok{"descarga\_original.xlsx"}\NormalTok{)) }\SpecialCharTok{|}\ErrorTok{\textgreater{}}
               \FunctionTok{clean\_names}\NormalTok{()}
\CommentTok{\#\textgreater{} New names:}
\CommentTok{\#\textgreater{} * \textasciigrave{}\textasciigrave{} {-}\textgreater{} ...100}
\CommentTok{\#\textgreater{} * \textasciigrave{}\textasciigrave{} {-}\textgreater{} ...127}
\CommentTok{\#\textgreater{} * \textasciigrave{}\textasciigrave{} {-}\textgreater{} ...128}

\CommentTok{\# Módulo Identificación {-}{-}{-}{-}}

\NormalTok{df\_encuesta }\OtherTok{\textless{}{-}}\NormalTok{ df\_encuesta }\SpecialCharTok{|}\ErrorTok{\textgreater{}}
               \FunctionTok{rename}\NormalTok{(}\AttributeTok{dni =}\NormalTok{ por\_favor\_podrias\_ingresar\_tu\_dni,}
                      \AttributeTok{mail =}\NormalTok{ direccion\_de\_correo\_electronico) }\SpecialCharTok{|}\ErrorTok{\textgreater{}}
               \FunctionTok{relocate}\NormalTok{(marca\_temporal, }\AttributeTok{.after =} \FunctionTok{last\_col}\NormalTok{()) }\SpecialCharTok{|}\ErrorTok{\textgreater{}}
               \FunctionTok{relocate}\NormalTok{(dni) }\SpecialCharTok{|}\ErrorTok{\textgreater{}}
               \FunctionTok{mutate}\NormalTok{(}\AttributeTok{dni =} \FunctionTok{as.integer}\NormalTok{(dni))}

\CommentTok{\# M?dulo Demogr?fico Individual {-}{-}{-}{-}}

\NormalTok{df\_encuesta }\OtherTok{\textless{}{-}}\NormalTok{ df\_encuesta }\SpecialCharTok{|}\ErrorTok{\textgreater{}}
               \FunctionTok{rename}\NormalTok{(}\AttributeTok{sexo =}\NormalTok{ indique\_el\_sexo\_asignado\_al\_nacer,}
                      \AttributeTok{genero =}\NormalTok{ identidad\_de\_genero\_autopercibida\_en\_la\_actualidad,}
                      \AttributeTok{fecha\_nacimiento =}\NormalTok{ indique\_su\_fecha\_de\_nacimiento,}
                      \AttributeTok{estado\_civil =}\NormalTok{ actualmente\_usted\_esta) }\SpecialCharTok{|}\ErrorTok{\textgreater{}}
               \FunctionTok{mutate}\NormalTok{(}\AttributeTok{fecha\_nacimiento =} \FunctionTok{ymd}\NormalTok{(fecha\_nacimiento))}
\CommentTok{\#\textgreater{} Warning: All formats failed to parse. No formats found.}

\CommentTok{\# M?dulo Hogar, presencia de padres e hijes {-}{-}{-}{-}}

\NormalTok{df\_encuesta }\OtherTok{\textless{}{-}}\NormalTok{ df\_encuesta }\SpecialCharTok{|}\ErrorTok{\textgreater{}}
\FunctionTok{rename}\NormalTok{(}\AttributeTok{hog\_n\_miembros\_hogar =}\NormalTok{ contandose\_a\_usted\_mismo\_cuantas\_personas\_viven\_habitualmente\_en\_su\_hogar\_indique\_un\_numero,}
       \AttributeTok{hog\_convivencia\_padres =}\NormalTok{ actualmente\_vivis\_con\_algunos\_de\_tus\_papas\_mamas,}
       \AttributeTok{hog\_convivencia\_hijes =}\NormalTok{ actualmente\_tiene\_hijes\_conviviendo\_con\_usted)}
       
\CommentTok{\# M?dulo Cuidados del hogar. Ni?es y adultos mayores}

\NormalTok{df\_encuesta }\OtherTok{\textless{}{-}}\NormalTok{ df\_encuesta }\SpecialCharTok{|}\ErrorTok{\textgreater{}}
\FunctionTok{rename}\NormalTok{(}\AttributeTok{hog\_n\_menores\_6 =}\NormalTok{ de\_esos\_hijes\_cuantos\_tienen\_menos\_de\_6\_anos,}
       \AttributeTok{hog\_dificultad\_cuidados\_menores\_6 =}\NormalTok{ habitualmente\_tienen\_dificultades\_para\_organizar\_las\_tareas\_de\_cuidado\_de\_sus\_hijes,}
       \AttributeTok{hog\_principal\_dificultad\_menores\_6 =}\NormalTok{ cuales\_es\_la\_principal\_dificultad\_que\_tienen\_para\_cuidar\_a\_les\_nines\_menores\_de\_6\_anos,}
       \AttributeTok{hog\_mayores\_60 =}\NormalTok{ actualmente\_usted\_tiene\_adultos\_mayores\_mayores\_de\_60\_anos\_conviviendo\_con\_usted,}
       \AttributeTok{hog\_dificultad\_cuidados\_mayores\_60 =}\NormalTok{ habitualmente\_tienen\_dificultades\_para\_organizar\_las\_tareas\_de\_cuidado\_de\_personas\_adultas\_mayores,}
       \AttributeTok{hog\_principal\_dificultad\_mayores\_60 =}\NormalTok{ cuales\_es\_la\_principal\_dificultad\_que\_tienen\_para\_cuidar\_a\_los\_adultos\_mayores)}

\CommentTok{\# M?dulo Vivienda}

\NormalTok{df\_encuesta }\OtherTok{\textless{}{-}}\NormalTok{ df\_encuesta }\SpecialCharTok{|}\ErrorTok{\textgreater{}}
\FunctionTok{rename}\NormalTok{(}
\AttributeTok{viv\_calle =}\NormalTok{ la\_calle\_de\_tu\_vivienda\_es,}
\AttributeTok{viv\_altura =}\NormalTok{ el\_numero\_de\_la\_altura\_de\_la\_calle\_es,}
\AttributeTok{viv\_partido =}\NormalTok{ esa\_vivienda\_se\_encuentra\_en\_el\_partido\_de,}
\AttributeTok{viv\_basural =}\NormalTok{ la\_vivienda\_esta\_ubicada\_cerca\_de\_basural\_es\_3\_cuadras\_o\_menos,}
\AttributeTok{viv\_inundable =}\NormalTok{ la\_vivienda\_esta\_ubicada\_en\_zona\_inundable\_en\_los\_ultimos\_12\_meses,}
\AttributeTok{viv\_villa =}\NormalTok{ la\_vivienda\_esta\_ubicada\_en\_villa\_de\_emergencia\_y\_o\_asentamiento,}
\AttributeTok{viv\_tipo =}\NormalTok{ su\_vivienda\_podria\_ser\_clasificada\_como,}
\AttributeTok{viv\_piso\_interior =}\NormalTok{ los\_pisos\_interiores\_de\_esa\_vivienda\_son\_principalmente\_de,}
\AttributeTok{viv\_agua =}\NormalTok{ el\_agua\_de\_esa\_vivienda\_es\_de,}
\AttributeTok{viv\_bano =}\NormalTok{ el\_bano\_tiene,}
\AttributeTok{viv\_desague =}\NormalTok{ el\_desague\_del\_bano\_es,}
\AttributeTok{viv\_bano\_compartido =}\NormalTok{ el\_bano\_es\_de,}
\AttributeTok{viv\_n\_ambientes =}\NormalTok{ cuantos\_ambientes\_habitaciones\_tiene\_su\_vivienda\_excluyendo\_cocina\_bano\_pasillos\_lavadero\_y\_garage,}
\AttributeTok{viv\_n\_amb\_dormir =}\NormalTok{ de\_esos\_ambientes\_cuantos\_usan\_habitualmente\_para\_dormir,}
\AttributeTok{viv\_amb\_estudio =}\NormalTok{ utiliza\_alguno\_de\_los\_ambientes\_de\_la\_vivienda\_exclusivamente\_como\_lugar\_de\_estudio,}
\AttributeTok{viv\_apropiada\_estudio =}\NormalTok{ en\_general\_consideras\_que\_tu\_casa\_es\_un\_espacio\_apropiado\_para\_estudiar\_en\_tu\_casa,}
\AttributeTok{viv\_internet =}\NormalTok{ en\_esa\_vivienda\_tenes\_acceso\_a\_internet\_a\_traves\_de\_algun\_servicio\_por\_cable\_wifi\_etc\_que\_no\_sea\_el\_plan\_de\_datos\_de\_un\_celular,}
\AttributeTok{viv\_n\_celulares =}\NormalTok{ por\_favor\_no\_podrias\_indicar\_indicar\_el\_tipo\_y\_cantidad\_de\_dispositivos\_tecnologicos\_que\_existen\_en\_esa\_vivienda\_celular\_con\_plan\_de\_datos\_de\_internet,}
\AttributeTok{viv\_n\_tablets =}\NormalTok{ por\_favor\_no\_podrias\_indicar\_indicar\_el\_tipo\_y\_cantidad\_de\_dispositivos\_tecnologicos\_que\_existen\_en\_esa\_vivienda\_tablet,}
\AttributeTok{viv\_n\_notebook =}\NormalTok{ por\_favor\_no\_podrias\_indicar\_indicar\_el\_tipo\_y\_cantidad\_de\_dispositivos\_tecnologicos\_que\_existen\_en\_esa\_vivienda\_notebook,}
\AttributeTok{viv\_n\_pc =}\NormalTok{ por\_favor\_no\_podrias\_indicar\_indicar\_el\_tipo\_y\_cantidad\_de\_dispositivos\_tecnologicos\_que\_existen\_en\_esa\_vivienda\_pc\_de\_escritorio,}
\AttributeTok{viv\_uso\_cocina =}\NormalTok{ para\_cocinar\_utiliza\_principalmente,}
\AttributeTok{viv\_relacion\_legal =}\NormalTok{ por\_ultimo\_este\_hogar\_es) }\SpecialCharTok{|}\ErrorTok{\textgreater{}}
\FunctionTok{relocate}\NormalTok{(}\FunctionTok{c}\NormalTok{(}\StringTok{"viv\_n\_celulares"}\NormalTok{,}
           \StringTok{"viv\_n\_tablets"}\NormalTok{,}
           \StringTok{"viv\_n\_notebook"}\NormalTok{,}
           \StringTok{"viv\_n\_pc"}\NormalTok{), }\AttributeTok{.after =}\NormalTok{ viv\_internet) }\SpecialCharTok{|}\ErrorTok{\textgreater{}}
\FunctionTok{relocate}\NormalTok{(viv\_bano\_compartido, }\AttributeTok{.after =}\NormalTok{ viv\_desague)}

\CommentTok{\# M?dulo Uso del tiempo {-}{-}{-}{-}}

\NormalTok{df\_encuesta }\OtherTok{\textless{}{-}}\NormalTok{ df\_encuesta }\SpecialCharTok{|}\ErrorTok{\textgreater{}}
\FunctionTok{rename}\NormalTok{(}\AttributeTok{tiempo\_limpiar\_casa =}\NormalTok{ durante\_la\_semana\_pasada\_hizo\_alguna\_de\_las\_siguientes\_actividades\_de\_la\_casa\_limpiar\_y\_ordenar\_la\_casa,}
\AttributeTok{tiempo\_planchar =}\NormalTok{ durante\_la\_semana\_pasada\_hizo\_alguna\_de\_las\_siguientes\_actividades\_de\_la\_casa\_planchar,}
\AttributeTok{tiempo\_comida =}\NormalTok{ durante\_la\_semana\_pasada\_hizo\_alguna\_de\_las\_siguientes\_actividades\_de\_la\_casa\_hacer\_la\_comida,}
\AttributeTok{tiempo\_refaccion =}\NormalTok{ durante\_la\_semana\_pasada\_hizo\_alguna\_de\_las\_siguientes\_actividades\_de\_la\_casa\_tareas\_de\_construccion\_o\_refaccion\_de\_la\_vivienda,}
\AttributeTok{tiempo\_cuidado\_menores =}\NormalTok{ durante\_la\_semana\_pasada\_hizo\_alguna\_de\_las\_siguientes\_actividades\_de\_la\_casa\_cuidar\_a\_los\_as\_nino\_as\_o\_hermanos\_as\_menores,}
\AttributeTok{tiempo\_cuidado\_mayores =}\NormalTok{ durante\_la\_semana\_pasada\_hizo\_alguna\_de\_las\_siguientes\_actividades\_de\_la\_casa\_cuidar\_a\_discapacitados\_o\_adultos\_mayores,}
\AttributeTok{tiempo\_compras =}\NormalTok{ durante\_la\_semana\_pasada\_hizo\_alguna\_de\_las\_siguientes\_actividades\_de\_la\_casa\_hacer\_las\_compras) }\SpecialCharTok{|}\ErrorTok{\textgreater{}}
\FunctionTok{relocate}\NormalTok{(}\FunctionTok{c}\NormalTok{(}\StringTok{"tiempo\_limpiar\_casa"}\NormalTok{,}
           \StringTok{"tiempo\_planchar"}\NormalTok{,}
           \StringTok{"tiempo\_comida"}\NormalTok{,}
           \StringTok{"tiempo\_refaccion"}\NormalTok{,}
           \StringTok{"tiempo\_cuidado\_menores"}\NormalTok{,}
           \StringTok{"tiempo\_cuidado\_mayores"}\NormalTok{,}
           \StringTok{"tiempo\_compras"}\NormalTok{), }\AttributeTok{.after =}\NormalTok{ viv\_relacion\_legal)}

\CommentTok{\# M?dulo Inseguridad Alimentaria {-}{-}{-}{-}}

\NormalTok{df\_encuesta }\OtherTok{\textless{}{-}}\NormalTok{ df\_encuesta }\SpecialCharTok{|}\ErrorTok{\textgreater{}}
\FunctionTok{rename}\NormalTok{(}
\AttributeTok{seg\_alim\_preocupacion\_falta\_alimentos =}\NormalTok{ usted\_se\_haya\_preocupado\_por\_no\_tener\_suficientes\_alimentos\_para\_comer\_por\_falta\_de\_dinero\_u\_otros\_recursos,}
\AttributeTok{seg\_alim\_falta\_alimentos\_nutritivos =}\NormalTok{ pensando\_aun\_en\_los\_ultimos\_12\_meses\_hubo\_alguna\_vez\_en\_que\_usted\_no\_haya\_podido\_comer\_alimentos\_saludables\_y\_nutritivos\_por\_falta\_de\_dinero\_u\_otros\_recursos,}
\AttributeTok{seg\_alim\_poca\_variedad =}\NormalTok{ hubo\_alguna\_vez\_en\_que\_usted\_haya\_comido\_poca\_variedad\_de\_alimentos\_por\_falta\_de\_dinero\_u\_otros\_recursos,}
\AttributeTok{seg\_alim\_falta\_una\_comida =}\NormalTok{ hubo\_alguna\_vez\_en\_que\_usted\_haya\_tenido\_que\_dejar\_de\_desayunar\_almorzar\_o\_cenar\_porque\_no\_habia\_suficiente\_dinero\_u\_otros\_recursos\_para\_obtener\_alimentos,}
\AttributeTok{seg\_alim\_comer\_menos =}\NormalTok{ pensando\_aun\_en\_los\_ultimos\_12\_meses\_hubo\_alguna\_vez\_en\_que\_usted\_haya\_comido\_menos\_de\_lo\_que\_pensaba\_que\_debia\_comer\_por\_falta\_de\_dinero\_u\_otros\_recursos,}
\AttributeTok{seg\_alim\_sin\_alimentos\_hogar =}\NormalTok{ hubo\_alguna\_vez\_en\_que\_su\_hogar\_se\_haya\_quedado\_sin\_alimentos\_por\_falta\_de\_dinero\_u\_otros\_recursos,}
\AttributeTok{seg\_alim\_sentir\_hambre =}\NormalTok{ hubo\_alguna\_vez\_en\_que\_usted\_haya\_sentido\_hambre\_pero\_no\_comio\_porque\_no\_habia\_suficiente\_dinero\_u\_otros\_recursos\_para\_obtener\_alimentos,}
\AttributeTok{seg\_alim\_sin\_comer\_todo\_un\_dia =}\NormalTok{ hubo\_alguna\_vez\_en\_que\_usted\_haya\_dejado\_de\_comer\_todo\_un\_dia\_por\_falta\_de\_dinero\_u\_otros\_recursos}
\NormalTok{)}
          
\CommentTok{\# M?dulo Preferencias Sociales {-}{-}{-}{-}}

\NormalTok{df\_encuesta }\OtherTok{\textless{}{-}}\NormalTok{ df\_encuesta }\SpecialCharTok{|}\ErrorTok{\textgreater{}}
\FunctionTok{rename}\NormalTok{(}
\AttributeTok{pref\_soc\_castigar\_trato\_injusto\_personal =}\NormalTok{ que\_tan\_dispuesto\_esta\_usted\_a\_castigar\_a\_alguien\_que\_lo\_a\_tratado\_a\_a\_usted\_injustamente\_incluso\_cuando\_existan\_riesgos\_para\_usted\_de\_sufrir\_consecuencias\_personales,}
\AttributeTok{pref\_soc\_castigar\_trato\_injusto\_terceros =}\NormalTok{ que\_tan\_dispuesto\_a\_esta\_usted\_a\_castigar\_a\_alguien\_que\_trata\_a\_los\_demas\_injustamente\_incluso\_cuando\_existan\_riesgos\_para\_usted\_de\_sufrir\_consecuencias\_personales,}
\AttributeTok{pref\_soc\_donacion\_benefica =}\NormalTok{ que\_tan\_dispuesto\_a\_esta\_usted\_a\_hacer\_donaciones\_a\_causas\_beneficas\_sin\_esperar\_nada\_a\_cambio,}
\AttributeTok{pref\_soc\_devolucion\_favores =}\NormalTok{ cuando\_alguien\_me\_hace\_un\_favor\_estoy\_dispuesto\_a\_devolverlo,}
\AttributeTok{pref\_soc\_revancha\_injusto\_personal =}\NormalTok{ si\_me\_tratan\_muy\_injustamente\_tomare\_revancha\_en\_la\_primera\_ocasion\_incluso\_aunque\_deba\_pagar\_un\_costo\_por\_ello,}
\AttributeTok{pref\_soc\_creencia\_buenas\_inteciones =}\NormalTok{ supongo\_que\_la\_gente\_tiene\_solo\_las\_mejores\_intenciones,}
\AttributeTok{pref\_soc\_extrano\_obsequio =}\NormalTok{ le\_daria\_al\_extrano\_uno\_de\_los\_obsequios\_como\_agradecimiento,}
\AttributeTok{pref\_soc\_monto\_donacion =}\NormalTok{ imaginese\_la\_siguiente\_situacion\_hoy\_de\_forma\_inesperada\_usted\_recibe\_15\_000\_pesos\_argentinos\_que\_cantidad\_de\_ese\_monto\_donaria\_usted\_a\_una\_buena\_causa)}

\CommentTok{\# M?dulo Origen Social {-}{-}{-}{-}{-}}

\NormalTok{df\_encuesta }\OtherTok{\textless{}{-}}\NormalTok{ df\_encuesta }\SpecialCharTok{|}\ErrorTok{\textgreater{}}
\FunctionTok{rename}\NormalTok{(}
\AttributeTok{os\_psh =}\NormalTok{ cuando\_usted\_tenia\_16\_anos\_quien\_era\_el\_principal\_sosten\_de\_su\_hogar\_psh\_la\_persona\_que\_realizaba\_el\_mayor\_aporte\_economico\_al\_hogar,}
\AttributeTok{os\_psh\_ne =}\NormalTok{ cual\_era\_el\_nivel\_educativo\_de\_esa\_persona,}
\AttributeTok{os\_psh\_ocupacion =}\NormalTok{ cual\_era\_la\_ocupacion\_principal\_de\_esta\_persona\_cuando\_usted\_tenia\_alrededor\_de\_16\_anos,}
\AttributeTok{os\_psh\_tarea =}\NormalTok{ que\_tareas\_hacia\_esa\_persona\_en\_ese\_trabajo,}
\AttributeTok{os\_psh\_jerarquia =}\NormalTok{ formaba\_parte\_del\_empleo\_del\_psh\_supervisar\_el\_trabajo\_de\_otros\_o\_decirles\_que\_hacer,}
\AttributeTok{os\_psh\_cat\_ocup =}\NormalTok{ en\_ese\_trabajo\_esa\_persona\_era,}
\AttributeTok{os\_psh\_cant\_trabajadores =}\NormalTok{ incluido\_el\_psh\_cuantas\_personas\_trabajaban\_en\_ese\_establecimiento,}
\AttributeTok{os\_conyuge =}\NormalTok{ cuando\_usted\_tenia\_16\_anos\_quien\_era\_el\_conyuge\_del\_psh\_de\_su\_hogar,}
\AttributeTok{os\_conyuge\_ne =}\NormalTok{ cual\_era\_el\_nivel\_educativo\_del\_conyuge\_del\_psh)}

\CommentTok{\# M?dulo Trabajo actual {-}{-}{-}{-}}

\NormalTok{df\_encuesta }\OtherTok{\textless{}{-}}\NormalTok{ df\_encuesta }\SpecialCharTok{|}\ErrorTok{\textgreater{}}
\FunctionTok{rename}\NormalTok{(}
\AttributeTok{trab\_trabajo =}\NormalTok{ entendiendo\_por\_trabajo\_a\_una\_actividad\_que\_genera\_bienes\_o\_servicios\_para\_el\_mercado\_y\_que\_se\_recibe\_a\_cambio\_dinero\_o\_especies\_la\_semana\_pasada\_usted\_trabajo\_por\_lo\_menos\_una\_hora,}
\AttributeTok{trab\_busco =}\NormalTok{ busco\_trabajo\_en\_los\_ultimos\_30\_dias\_o\_esta\_tratando\_de\_ponerse\_alguno\_por\_su\_cuenta,}
\AttributeTok{trab\_tiempo\_busco =}\NormalTok{ cuanto\_hace\_que\_esta\_en\_la\_situacion\_anterior,}
\AttributeTok{trab\_ocupacion =}\NormalTok{ como\_se\_llama\_la\_ocupacion\_de\_la\_que\_usted\_trabajo\_la\_semana\_pasada,}
\AttributeTok{trab\_tarea =}\NormalTok{ que\_tareas\_realiza\_en\_ese\_trabajo,}
\AttributeTok{trab\_jerarquia =}\NormalTok{ forma\_parte\_de\_su\_empleo\_supervisar\_el\_trabajo\_de\_otros\_o\_decirles\_que\_hacer,}
\AttributeTok{trab\_cat\_ocup =}\NormalTok{ en\_ese\_trabajo\_usted\_es,}
\AttributeTok{trab\_cant\_trabajadores =}\NormalTok{ incluido\_usted\_cuantas\_personas\_trabajan\_en\_ese\_establecimiento,}
\AttributeTok{trab\_jubilacion =}\NormalTok{ en\_ese\_trabajo,}
\AttributeTok{trab\_vacaciones =}\NormalTok{ en\_ese\_trabajo\_usted\_tiene\_vacaciones\_pagas,}
\AttributeTok{trab\_obra\_social =}\NormalTok{ en\_ese\_trabajo\_usted\_tiene\_obra\_social\_prepaga,}
\AttributeTok{trab\_aguinaldo =}\NormalTok{ en\_ese\_trabajo\_usted\_tiene\_aguinaldo,}
\AttributeTok{trab\_salario\_familiar =}\NormalTok{ en\_ese\_trabajo\_usted\_tiene\_salario\_familiar,}
\AttributeTok{trab\_dias\_enfermedad =}\NormalTok{ en\_ese\_trabajo\_usted\_tiene\_dias\_pagos\_por\_enfermedad,}
\AttributeTok{trab\_dias\_estudio =}\NormalTok{ en\_ese\_trabajo\_usted\_tiene\_dias\_por\_estudio,}
\AttributeTok{trab\_horas =}\NormalTok{ por\_ultimo\_cuantas\_horas\_incluyendo\_las\_horas\_extras\_trabajo\_usted\_la\_semana\_pasada)}

\CommentTok{\# M?dulo Ingresos del Hogar}

\NormalTok{df\_encuesta }\OtherTok{\textless{}{-}}\NormalTok{ df\_encuesta }\SpecialCharTok{|}\ErrorTok{\textgreater{}}
\FunctionTok{rename}\NormalTok{(}
\AttributeTok{ing\_trabajo =}\NormalTok{ teniendo\_en\_cuenta\_no\_solo\_sus\_propios\_ingresos\_sino\_tambien\_los\_de\_las\_demas\_personas\_de\_su\_hogar\_indique\_cuales\_de\_los\_siguientes\_tipos\_de\_ingresos\_tuvieron\_en\_su\_hogar\_durante\_el\_mes\_pasado\_lo\_que\_ganan\_por\_trabajo,}
\AttributeTok{ing\_programas\_empleo =}\NormalTok{ teniendo\_en\_cuenta\_no\_solo\_sus\_propios\_ingresos\_sino\_tambien\_los\_de\_las\_demas\_personas\_de\_su\_hogar\_indique\_cuales\_de\_los\_siguientes\_tipos\_de\_ingresos\_tuvieron\_en\_su\_hogar\_durante\_el\_mes\_pasado\_programas\_de\_empleo,}
\AttributeTok{ing\_jubilacion =}\NormalTok{ teniendo\_en\_cuenta\_no\_solo\_sus\_propios\_ingresos\_sino\_tambien\_los\_de\_las\_demas\_personas\_de\_su\_hogar\_indique\_cuales\_de\_los\_siguientes\_tipos\_de\_ingresos\_tuvieron\_en\_su\_hogar\_durante\_el\_mes\_pasado\_jubilacion,}
\AttributeTok{ing\_auh =}\NormalTok{ teniendo\_en\_cuenta\_no\_solo\_sus\_propios\_ingresos\_sino\_tambien\_los\_de\_las\_demas\_personas\_de\_su\_hogar\_indique\_cuales\_de\_los\_siguientes\_tipos\_de\_ingresos\_tuvieron\_en\_su\_hogar\_durante\_el\_mes\_pasado\_asignacion\_universal\_por\_hijo\_auh,}
\AttributeTok{ing\_pension =}\NormalTok{ teniendo\_en\_cuenta\_no\_solo\_sus\_propios\_ingresos\_sino\_tambien\_los\_de\_las\_demas\_personas\_de\_su\_hogar\_indique\_cuales\_de\_los\_siguientes\_tipos\_de\_ingresos\_tuvieron\_en\_su\_hogar\_durante\_el\_mes\_pasado\_algun\_otro\_tipo\_de\_pension\_especifica\_7\_hijos\_invalidez\_o\_discapacidad\_etc,}
\AttributeTok{ing\_indenmizacion =}\NormalTok{ teniendo\_en\_cuenta\_no\_solo\_sus\_propios\_ingresos\_sino\_tambien\_los\_de\_las\_demas\_personas\_de\_su\_hogar\_indique\_cuales\_de\_los\_siguientes\_tipos\_de\_ingresos\_tuvieron\_en\_su\_hogar\_durante\_el\_mes\_pasado\_indemnizacion\_por\_despido,}
\AttributeTok{ing\_seguro\_desempleo =}\NormalTok{ teniendo\_en\_cuenta\_no\_solo\_sus\_propios\_ingresos\_sino\_tambien\_los\_de\_las\_demas\_personas\_de\_su\_hogar\_indique\_cuales\_de\_los\_siguientes\_tipos\_de\_ingresos\_tuvieron\_en\_su\_hogar\_durante\_el\_mes\_pasado\_seguro\_de\_desempleo,}
\AttributeTok{ing\_alquiler =}\NormalTok{ teniendo\_en\_cuenta\_no\_solo\_sus\_propios\_ingresos\_sino\_tambien\_los\_de\_las\_demas\_personas\_de\_su\_hogar\_indique\_cuales\_de\_los\_siguientes\_tipos\_de\_ingresos\_tuvieron\_en\_su\_hogar\_durante\_el\_mes\_pasado\_alquiler\_de\_una\_propiedad,}
\AttributeTok{ing\_beca =}\NormalTok{ teniendo\_en\_cuenta\_no\_solo\_sus\_propios\_ingresos\_sino\_tambien\_los\_de\_las\_demas\_personas\_de\_su\_hogar\_indique\_cuales\_de\_los\_siguientes\_tipos\_de\_ingresos\_tuvieron\_en\_su\_hogar\_durante\_el\_mes\_pasado\_beca\_de\_estudios,}
\AttributeTok{ing\_cuota\_alimentaria =}\NormalTok{ teniendo\_en\_cuenta\_no\_solo\_sus\_propios\_ingresos\_sino\_tambien\_los\_de\_las\_demas\_personas\_de\_su\_hogar\_indique\_cuales\_de\_los\_siguientes\_tipos\_de\_ingresos\_tuvieron\_en\_su\_hogar\_durante\_el\_mes\_pasado\_cuota\_alimentaria,}
\AttributeTok{ing\_n\_aportantes =}\NormalTok{ contandose\_usted\_el\_mes\_pasado\_cuantas\_personas\_que\_viven\_en\_este\_hogar\_aportaron\_ingresos\_monetarios,}
\AttributeTok{ing\_totales =}\NormalTok{ sumando\_lo\_aportado\_por\_todos\_los\_miembros\_del\_hogar\_mas\_lo\_que\_recibieron\_de\_otras\_partes\_como\_ayudas\_de\_otras\_personas\_planes\_sociales\_seguro\_de\_desempleo\_u\_otros\_ingresos\_cuanto\_fue\_el\_ingreso\_en\_dinero\_total\_del\_hogar\_el\_mes\_pasado)}

\CommentTok{\# M?dulo Acad?mico UNAJ {-}{-}{-}{-}}

\NormalTok{df\_encuesta }\OtherTok{\textless{}{-}}\NormalTok{ df\_encuesta }\SpecialCharTok{|}\ErrorTok{\textgreater{}}
\FunctionTok{rename}\NormalTok{(}
\AttributeTok{unaj\_ano\_ingreso =}\NormalTok{ en\_que\_ano\_ingresaste\_a\_la\_unaj,}
\AttributeTok{unaj\_estudios\_anteriores =}\NormalTok{ cursaste\_estudios\_terciarios\_universitarios\_antes\_de\_ingresar\_al\_unaj,}
\AttributeTok{unaj\_n\_materias\_cursando =}\NormalTok{ cuantas\_materias\_estas\_cursando\_este\_cuatrimestre,}
\AttributeTok{unaj\_periodo\_lectivo =}\NormalTok{ indica\_tu\_actual\_periodo\_lectivo\_primer\_o\_el\_segundo\_cuatrimestre,}
\AttributeTok{unaj\_ano\_lectivo =}\NormalTok{ indica\_tu\_actual\_ano\_lectivo,}
\AttributeTok{unaj\_horas\_estudio =}\NormalTok{ pensando\_en\_este\_cuatrimestre\_cuantas\_horas\_semanales\_incluyendo\_las\_horas\_de\_la\_propia\_cursada\_le\_estas\_dedicando\_en\_promedio\_al\_estudio,}
\AttributeTok{unaj\_n\_materias\_aprobadas =}\NormalTok{ cuantas\_materias\_aprobadas\_de\_la\_carrera\_tenes,}
\AttributeTok{unaj\_progresar =}\NormalTok{ actualmente\_sos\_beneficiario\_del\_progresar) }\SpecialCharTok{|}\ErrorTok{\textgreater{}}
\FunctionTok{relocate}\NormalTok{(}\FunctionTok{c}\NormalTok{(}\StringTok{"unaj\_ano\_lectivo"}\NormalTok{,}
           \StringTok{"unaj\_periodo\_lectivo"}\NormalTok{,}
           \StringTok{"unaj\_horas\_estudio"}\NormalTok{), }\AttributeTok{.after =}\NormalTok{ unaj\_n\_materias\_cursando)}

\CommentTok{\# M?dulo materia Metodolog?a Cuantitativa}

\NormalTok{df\_encuesta }\OtherTok{\textless{}{-}}\NormalTok{ df\_encuesta }\SpecialCharTok{|}\ErrorTok{\textgreater{}}
\FunctionTok{rename}\NormalTok{(}
\AttributeTok{cuanti\_docente =}\NormalTok{ indica\_el\_docente\_de\_tu\_comision,}
\AttributeTok{cuanti\_1\_palabra =}\NormalTok{ primer\_palabra,}
\AttributeTok{cuanti\_2\_palabra =}\NormalTok{ segunda\_palabra,}
\AttributeTok{cuanti\_3\_palabra =}\NormalTok{ tercera\_palabra)}

\CommentTok{\# M?dulo Redes Sociales}

\NormalTok{df\_encuesta }\OtherTok{\textless{}{-}}\NormalTok{ df\_encuesta }\SpecialCharTok{|}\ErrorTok{\textgreater{}}
\FunctionTok{rename}\NormalTok{(}
\AttributeTok{redes\_tu\_nombre =}\NormalTok{ por\_favor\_selecciona\_tu\_nombre\_dentro\_de\_esta\_lista,}
\AttributeTok{redes\_1\_contacto =}\NormalTok{ selecciona\_al\_primer\_estudiante,}
\AttributeTok{redes\_2\_contacto =}\NormalTok{ selecciona\_al\_segundo\_estudiante,}
\AttributeTok{redes\_3\_contacto =}\NormalTok{ selecciona\_al\_tercer\_estudiante,}
\AttributeTok{redes\_4\_contacto =}\NormalTok{ selecciona\_al\_cuarto\_estudiante,}
\AttributeTok{redes\_5\_contacto =}\NormalTok{ selecciona\_al\_quinto\_estudiante) }\SpecialCharTok{|}\ErrorTok{\textgreater{}}
\FunctionTok{relocate}\NormalTok{(redes\_tu\_nombre, }\AttributeTok{.before =}\NormalTok{ redes\_1\_contacto)}

\CommentTok{\# Grabo el archivo}

\FunctionTok{write\_rds}\NormalTok{(df\_encuesta,}
          \FunctionTok{here}\NormalTok{(}\StringTok{"Outputs"}\NormalTok{,}\StringTok{"df\_encuesta.rds"}\NormalTok{))}
                  
                  
\end{Highlighting}
\end{Shaded}

\hypertarget{parts}{%
\chapter{Parts}\label{parts}}

You can add parts to organize one or more book chapters together. Parts can be inserted at the top of an .Rmd file, before the first-level chapter heading in that same file.

Add a numbered part: \texttt{\#\ (PART)\ Act\ one\ \{-\}} (followed by \texttt{\#\ A\ chapter})

Add an unnumbered part: \texttt{\#\ (PART\textbackslash{}*)\ Act\ one\ \{-\}} (followed by \texttt{\#\ A\ chapter})

Add an appendix as a special kind of un-numbered part: \texttt{\#\ (APPENDIX)\ Other\ stuff\ \{-\}} (followed by \texttt{\#\ A\ chapter}). Chapters in an appendix are prepended with letters instead of numbers.

\hypertarget{footnotes-and-citations}{%
\chapter{Footnotes and citations}\label{footnotes-and-citations}}

\hypertarget{footnotes}{%
\section{Footnotes}\label{footnotes}}

Footnotes are put inside the square brackets after a caret \texttt{\^{}{[}{]}}. Like this one \footnote{This is a footnote.}.

\hypertarget{citations}{%
\section{Citations}\label{citations}}

Reference items in your bibliography file(s) using \texttt{@key}.

For example, we are using the \textbf{bookdown} package \citep{R-bookdown} (check out the last code chunk in index.Rmd to see how this citation key was added) in this sample book, which was built on top of R Markdown and \textbf{knitr} \citep{xie2015} (this citation was added manually in an external file book.bib).
Note that the \texttt{.bib} files need to be listed in the index.Rmd with the YAML \texttt{bibliography} key.

The \texttt{bs4\_book} theme makes footnotes appear inline when you click on them. In this example book, we added \texttt{csl:\ chicago-fullnote-bibliography.csl} to the \texttt{index.Rmd} YAML, and include the \texttt{.csl} file. To download a new style, we recommend: \url{https://www.zotero.org/styles/}

The RStudio Visual Markdown Editor can also make it easier to insert citations: \url{https://rstudio.github.io/visual-markdown-editing/\#/citations}

\hypertarget{blocks}{%
\chapter{Blocks}\label{blocks}}

\hypertarget{equations}{%
\section{Equations}\label{equations}}

Here is an equation.

\begin{equation} 
  f\left(k\right) = \binom{n}{k} p^k\left(1-p\right)^{n-k}
  \label{eq:binom}
\end{equation}

You may refer to using \texttt{\textbackslash{}@ref(eq:binom)}, like see Equation \eqref{eq:binom}.

\hypertarget{theorems-and-proofs}{%
\section{Theorems and proofs}\label{theorems-and-proofs}}

Labeled theorems can be referenced in text using \texttt{\textbackslash{}@ref(thm:tri)}, for example, check out this smart theorem \ref{thm:tri}.

\begin{theorem}
\protect\hypertarget{thm:tri}{}\label{thm:tri}For a right triangle, if \(c\) denotes the \emph{length} of the hypotenuse
and \(a\) and \(b\) denote the lengths of the \textbf{other} two sides, we have
\[a^2 + b^2 = c^2\]
\end{theorem}

Read more here \url{https://bookdown.org/yihui/bookdown/markdown-extensions-by-bookdown.html}.

\hypertarget{callout-blocks}{%
\section{Callout blocks}\label{callout-blocks}}

The \texttt{bs4\_book} theme also includes special callout blocks, like this \texttt{.rmdnote}.

You can use \textbf{markdown} inside a block.

\begin{Shaded}
\begin{Highlighting}[]
\FunctionTok{head}\NormalTok{(beaver1, }\AttributeTok{n =} \DecValTok{5}\NormalTok{)}
\CommentTok{\#\textgreater{}   day time  temp activ}
\CommentTok{\#\textgreater{} 1 346  840 36.33     0}
\CommentTok{\#\textgreater{} 2 346  850 36.34     0}
\CommentTok{\#\textgreater{} 3 346  900 36.35     0}
\CommentTok{\#\textgreater{} 4 346  910 36.42     0}
\CommentTok{\#\textgreater{} 5 346  920 36.55     0}
\end{Highlighting}
\end{Shaded}

It is up to the user to define the appearance of these blocks for LaTeX output.

You may also use: \texttt{.rmdcaution}, \texttt{.rmdimportant}, \texttt{.rmdtip}, or \texttt{.rmdwarning} as the block name.

The R Markdown Cookbook provides more help on how to use custom blocks to design your own callouts: \url{https://bookdown.org/yihui/rmarkdown-cookbook/custom-blocks.html}

\hypertarget{sharing-your-book}{%
\chapter{Sharing your book}\label{sharing-your-book}}

\hypertarget{publishing}{%
\section{Publishing}\label{publishing}}

HTML books can be published online, see: \url{https://bookdown.org/yihui/bookdown/publishing.html}

\hypertarget{pages}{%
\section{404 pages}\label{pages}}

By default, users will be directed to a 404 page if they try to access a webpage that cannot be found. If you'd like to customize your 404 page instead of using the default, you may add either a \texttt{\_404.Rmd} or \texttt{\_404.md} file to your project root and use code and/or Markdown syntax.

\hypertarget{metadata-for-sharing}{%
\section{Metadata for sharing}\label{metadata-for-sharing}}

Bookdown HTML books will provide HTML metadata for social sharing on platforms like Twitter, Facebook, and LinkedIn, using information you provide in the \texttt{index.Rmd} YAML. To setup, set the \texttt{url} for your book and the path to your \texttt{cover-image} file. Your book's \texttt{title} and \texttt{description} are also used.

This \texttt{bs4\_book} provides enhanced metadata for social sharing, so that each chapter shared will have a unique description, auto-generated based on the content.

Specify your book's source repository on GitHub as the \texttt{repo} in the \texttt{\_output.yml} file, which allows users to view each chapter's source file or suggest an edit. Read more about the features of this output format here:

\url{https://pkgs.rstudio.com/bookdown/reference/bs4_book.html}

Or use:

\begin{Shaded}
\begin{Highlighting}[]
\NormalTok{?bookdown}\SpecialCharTok{::}\NormalTok{bs4\_book}
\end{Highlighting}
\end{Shaded}


  \bibliography{book.bib,packages.bib}

\end{document}
